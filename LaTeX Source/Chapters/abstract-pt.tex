%!TEX root = ../template.tex
%%%%%%%%%%%%%%%%%%%%%%%%%%%%%%%%%%%%%%%%%%%%%%%%%%%%%%%%%%%%%%%%%%%%
%% abstrac-pt.tex
%% NOVA thesis document file
%%
%% Abstract in Portuguese
%%%%%%%%%%%%%%%%%%%%%%%%%%%%%%%%%%%%%%%%%%%%%%%%%%%%%%%%%%%%%%%%%%%%



Desenvolvimentos recentes em dispositivos IoT e em sistemas de comunicação como o 5G trouxeram novas soluções capazes de oferecer uma deteção avançada dos ambientes circundantes. Por outro lado, a esperança média de vida aumentou, levando a um aumento considerável do número de pessoas idosas. Consequentemente, existe uma procura constante de soluções de suporte a uma Vida Ativa e Assistida dessas pessoas. Esta tese pretende propor uma solução que ajude a conhecer a localização dos dispositivos IoT, que possam estar a ajudar as pessoas. A solução proposta deve ter em consideração os fatores de risco, das pessoas-alvo em cada momento e também as restrições técnicas do dispositivo, como a energia disponível e os meios de comunicação. Assim, uma decisão baseada num perfil deve ser tomada autonomamente pelo dispositivo ou pelo seu sistema, para garantir a utilização da melhor tecnologia de geolocalização em cada situação.





% Palavras-chave do resumo em Português
\begin{keywords}
VAA, IoT, Geolocalização
\end{keywords}
% to add an extra black line
