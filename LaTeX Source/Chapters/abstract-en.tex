%!TEX root = ../template.tex
%%%%%%%%%%%%%%%%%%%%%%%%%%%%%%%%%%%%%%%%%%%%%%%%%%%%%%%%%%%%%%%%%%%%
%% abstrac-en.tex
%% NOVA thesis document file
%%
%% Abstract in English
%%%%%%%%%%%%%%%%%%%%%%%%%%%%%%%%%%%%%%%%%%%%%%%%%%%%%%%%%%%%%%%%%%%%
   
   


 Recent developments in IoT devices and in communication systems as 5G has brought new solutions capable of offering advanced sensing of the surrounding environments. On the other hand, the average life expectancy has augmented leading to a considerable increase in the number of elderly people. Consequently, there is a constant demand for solutions to support an Active and Assisted Living (AAL) of such people. This thesis intends to propose a solution to help in knowing the location of IoT devices that could be assisting the people. The proposed solution should take into consideration the risk factors of the target people at each moment and as well the technical constraints of the device as available power energy and communications. Thus a profile-based decision should autonomously be made by the device or its integrated system to ensure the use of the best geolocation technology in each situation


% Palavras-chave do resumo em Inglês
\begin{keywords}
Active Assisted Living,IoT, Geolocation
\end{keywords} 